\documentclass[12pt, a4paper]{article}
\usepackage{polski}

\author{Krystian}
\title{Test}

\begin{document}
	$$
	\mathbf{X} =
	\left( \begin{array}{ccc}
	x_{11} & x_{12} & x_{13} \\
	x_{21} & x_{22} & x_{23} \\
	x_{35} & x_{34} & z_{33}
	\end{array} \right)
	+
	\left| \begin{array}{ccc}
	x_{11} & x_{12} & x_{13} \\
	x_{21} & x_{22} & x_{23} \\
	x_{35} & x_{34} & z_{33}
	\end{array} \right|
	$$ 
	
	$$
	y = \left\{ \begin{array}{ll}
	a & \textrm{gdy $d>c$}\\
	b+x & \textrm{gdy $d=c$}\\
	l & \textrm{gdy $ d < c $}
	\end{array} \right.
	$$
	
	\begin{tabular}{|r|l|} \hline %hline wstawia poziomą linię na szerokość tabeli
		7C0 & heksadecymalnie \\
		3700 & oktalnie \\
		11111000000 & binarnie \\
		\hline \hline
		1984 & dziesietnie \\ \hline
	\end{tabular}
		\\
	
		\begin{tabular}{|c|} \hline
			Ten akapit jest wewnątrz pudełka.
			Mamy nadzieję, że uzyskany
			efekt sie podoba.\\ \hline
		
	\end{tabular}
\end{document}